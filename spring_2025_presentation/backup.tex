
\documentclass{beamer}
\DeclareGraphicsExtensions{.eps, .pdf,.png,.jpg,.mps,}
\usepackage{graphicx}
\usepackage{epstopdf}
\usepackage{verbatim}
\usepackage[flushleft]{threeparttable}
\usepackage{caption}
\usepackage{hyperref}
\usepackage{subfig}
\usepackage{adjustbox}
\usepackage{overpic}
\usepackage{changepage}
\usepackage{lmodern}
\usepackage[utf8]{inputenc}
\usepackage[T1]{fontenc}
\usepackage{dcolumn}
\usepackage{booktabs}
\usepackage{threeparttable}
\usepackage[usenames,dvipsnames]{xcolor}
\usepackage{tikz} 
\usetikzlibrary{calc, arrows.meta, intersections, patterns, positioning, shapes.misc, fadings, through,decorations.pathreplacing}
\setbeamertemplate{itemize subitem}{\color{gray}\scriptsize$\bullet$}
\definecolor{ColorOne}{named}{MidnightBlue}
\definecolor{ColorTwo}{named}{Dandelion}
\definecolor{ColorThree}{named}{Plum}
\usepackage[dvipsnames]{xcolor}
\usepackage[table]{xcolor}
\usepackage{colortbl}
\usepackage{longtable}
\usepackage{tablefootnote}

% define colors
\definecolor{mypink1}{rgb}{0.858, 0.188, 0.478}
\definecolor{grn}{rgb}{0.0, 0.5, 0.0}
\definecolor{orn}{rgb}{1.0, 0.49, 0.0}
\definecolor{lightgreen}{RGB}{220, 255, 220}
\definecolor{medgreen}{RGB}{170, 230, 170}


% indentation
\renewcommand{\indent}{\hspace*{2em}}

\setlength{\leftmarginii}{3em}
\setbeamertemplate{itemize items}[circle]
\setbeamertemplate{footline}[frame number]
\usepackage{appendixnumberbeamer}

\usepackage[style=authoryear]{biblatex}
\bibliography{biblography}
\usepackage{soul}
\bfseries
\newcommand{\Blue}{\color[rgb]{0,0,1}}
\newcommand{\hilight}[1]{\colorbox{orange}{#1}}
\usepackage{soul}
\bfseries

\newenvironment{wideitemize}{\itemize\addtolength{\itemsep}{10pt}}{\enditemize}
\newenvironment{mfignotes}{\begin{tiny}\begin{minipage}{\textwidth}\begin{tiny}\smallskip\par Notes: }
{\end{tiny}\end{minipage}\end{tiny}}
\title []{CalFresh}

\author[Aman et al. (2025)] { Christine Aman, UC San Diego\\
Eric Floyd, UC San Diego \\
Mia Jerphagnon, UC San Diego \\
Sally Sadoff, UC San Diego}

\date{Fall, 2025}
\makeatletter
\setbeamertemplate{navigation symbols}{}
\begin{document}



\begin{frame}
 %\addtocounter{framenumber}{-1}
\titlepage
%\thispagestyle{empty}
\end{frame}

\begin{frame}{Motivations}
\begin{adjustwidth}{-2em}{-2em}
    \begin{wideitemize}
        \item A large body of research demonstrates that administrative burdens and complex processes deter enrollment in social programs ((e.g., Giannella et al., 2023))
        \item Growing evidence suggests that light-touch outreach and information about social programs may have a limited impact on takeup (e.g., Linos et al., 2022)
        \item There is evidence that more intensive outreach and assistance can be effective
        \item Literature also demonstrates the large impact of automatic enrollment on take up 
    \end{wideitemize}
\end{adjustwidth}
\end{frame}



\begin{frame}{Motivations}

% add more figures takeup, eligibility of snap nationally 
% 
\begin{adjustwidth}{-2em}{-2em}
    \begin{wideitemize}
        \item At UCSD, approximately 42\% of the undergraduate and 35\% of the graduate students have reported experiencing food insecurity
        \begin{itemize}
            \item The need is even more acute for underserved and underrepresented minority and Pell Grant-eligible  students, with rates rising to 53\% and 51\%, respectively (Carillo, 2024)
        \end{itemize}
        \item As of fall 2019, only 22\% and 29\% of eligible UC undergraduates and graduates were enrolled in CalFresh (California's version of the SNAP)
(Rothstein et al., 2024).
    \end{wideitemize}
\end{adjustwidth}
\end{frame}

\begin{frame}{Our Study}
% interest in actual enrollment process and the effects of receiving benefits 
% Two treatments 1 resolves the problem of administrative burden, provides relief while enrollment process stretches on, gives us a sense of what effect of automatic enrollment 
% link grocery card and link to automatic enrollment 
\begin{adjustwidth}{-2em}{-2em}
    \begin{wideitemize}
        \item Conduct a field experiment to test the impacts of personalized outreach and simulated automatic enrollment 
        \item These interventions address two primary barriers to CalFresh enrollment:
        \begin{itemize}
            \item Many students are deterred from enrolling in CalFresh due to its complicated application process 
            \item Students may not fully understand the benefits of
being on CalFresh
        \end{itemize}
        \item Examine the impact of personal outreach and step-by-step assistance with the enrollment process and of providing students with grocery cards that mimic the purchasing power of CalFresh EBT cards
    \end{wideitemize}
\end{adjustwidth}
\end{frame}

\begin{frame}{Experimental Design}
% "Grocery Card": auto enrollment mimic 
\begin{adjustwidth}{-2em}{-2em}
    \begin{wideitemize}
        \item {\bf{Control:}} Received an email stating that we had determined they were eligible to apply for Calfresh with a link to a webpage about the program

        \item {\bf{Assistance (Assist):}}  Aims to lower the administrative burden with high-touch individualized information and assistance throughout the enrollment process
        \begin{itemize}
            \item Assistants reach out to the participants receiving assistance via email and offer to help with the process of signing up via zoom or email 
\end{itemize}
\item \textbf{Grocery Card (GC):} Participants received a \$100 grocery card to the grocery store of their choice 
\item \textbf{Assist + GC:}The final treatment group received both the grocery card and assistance treatments

\end{wideitemize}
\end{adjustwidth}

\end{frame}

\begin{frame}{Timeline}
\begin{adjustwidth}{-2em}{-2em}
\tikzstyle{descript} = [text = black,align=left, minimum height=1.8cm, align=center, outer sep=0pt,font = \footnotesize]
\tikzstyle{activity} =[align=left,outer sep=1pt]

\begin{tikzpicture}[very thick, black]

%% Coordinates
\coordinate (O) at (-1.5,0); % Origin
\coordinate (P1) at (3,0);
\coordinate (P2) at (8,0);
\coordinate (P3) at (11,0);
\coordinate (F) at (13,0); %End

\coordinate (E2) at (3,0); %Event

%% Filled regions
\fill[color=ColorOne!20] rectangle (0) -- (P1) -- ($(P1)+(0,1)$) -- ($(0)+(0,1)$); 

\fill[color=ColorTwo!20] rectangle (P1) -- (P2) -- ($(P2)+(0,1)$) -- ($(P1)+(0,1)$); % Work
\fill[color=ColorThree!20] rectangle (P2) -- (P3) -- ($(P3)+(0,1)$) -- ($(P2)+(0,1)$); % Current work

%% Text inside filled regions
\draw ($(P1)+(-1.5,0.5)$) node[activity,ColorOne] {Enrollment \&\\ Baseline Survey};
\draw ($(P2)+(-2.5,0.5)$) node[activity,ColorTwo] {Outreach / Assistance};
\draw ($(P3)+(-1.5,0.5)$)  node[activity, ColorThree] {Endline Data \\Collection};




\draw [decorate,decoration={brace,amplitude=6pt}]($(E2)+(0,1.2)$) -- ($(E2)+(1,1.2)$) node [black,midway,above=6pt] {\scriptsize Grocery Card Selection / Disbursement};




\foreach \x in {0, 3, 8, 11}
\draw(\x cm,3pt) -- (\x cm,-3pt);
%% Labels
\foreach \i \j in {0/week 1, 3/week 4,8/week 9, 11/week 11,}{
\draw (\i,0) node[below=3pt] {\j} ;
}

\end{tikzpicture}
\end{adjustwidth}
\end{frame}


\begin{frame}{Participants}
% ADD IN THAT THIS IS WAVE 1
% ADD IN SAMPLE SIZE WHICH IS JUST THOSE WHO FILLED OUT ROI
% ADD TIMELINE BELOW
\begin{adjustwidth}{-2em}{-2em}
\begin{wideitemize}
    \item We enrolled only students who we determined to be eligible to \textit{apply} for CalFresh (who were not already currently enrolled)
    \begin{wideitemize}
        \item To be eligible to apply students must be U.S. citizens, be at least 18 and meet one of the following: \begin{enumerate}
            \item Approved and accepted a Federal Work Study award
            \item Received a TANF funded CalGrant A or B award
            \item Receiving California Work Opportunity and Responsibility to Kids (CalWORKs) or Tribal TANF
            \item Employed with an on campus job
        \end{enumerate}
    \end{wideitemize}
    \item Being eligible to apply for CalFresh does not necessarily mean that one is eligible for benefits 
\end{wideitemize}
\end{adjustwidth}
\end{frame}

\begin{frame}{Baseline Survey }
%BALANCE TABLE HERE 
\begin{adjustwidth}{-2em}{-2em}
The baseline survey collected information on:
    \begin{wideitemize}
        \item Eligibility to apply for Calfresh
        \item Food Insecurity 
        \begin{itemize}
            \item Measured by the U.S. Adult Food Security Survey Module
        \end{itemize}
        \item Mental health 
        \begin{itemize}
            \item Measured by GAD-7 and CES-D
        \end{itemize}
        \item Consumption of major food groups 
        \begin{itemize}
            \item Reported consumption of major food groups from the Dietary Guidelines for Americans, 
            \item Whether participants wants to increase, decrease, or maintain consumption of major food groups
        \end{itemize}
    \end{wideitemize}
The endline survey collected the same information with the following changes
\begin{wideitemize}
    \item No eligibility information was collected
    \item 
\end{wideitemize}

\end{adjustwidth}
\end{frame}


\begin{frame}{Distribution of Baseline Food Insecurity Score}
%ADD IN POPULATION DISTRIBUTION 
% ADD IN CATEGORIES 
\begin{center}
  \includegraphics[width=8cm, height=7cm]{afe_visuals/baseline_fi_dist.png}  
\end{center}

% DO OMNIBUS F TEST 
% NO LPIE 
% CHANGE TO BE AT RISK OF DEPRESSION INSTEAD OF JUST NUMBER 
    
\end{frame}


% \begin{frame}[t]{County Data: Application Rate by Treatment}
% \vspace*{1cm}
% \tiny
% \centering
% \begin{table}[htbp]
% \centering
% \begin{threeparttable}
% \caption{Application Rate by Treatment}
% \label{tab:outcomes}
% \begin{tabular}{l*{4}{c}c}
% \hline\hline
% & \multicolumn{1}{c}{Group 0} & \multicolumn{1}{c}{Group 1} & \multicolumn{1}{c}{Group 2} & \multicolumn{1}{c}{Group 3} & \multicolumn{1}{c}{Total} \\
% & \multicolumn{1}{c}{}  & \multicolumn{1}{c}{}  & \multicolumn{1}{c}{}  & \multicolumn{1}{c}{}  & \multicolumn{1}{c}{} \\
% \hline
% Did not apply           & 22 (75.86\%) & 20 (74.07\%) & 29 (78.38\%) & 21 (67.74\%) & 92 (74.19\%) \\
% Applied           &  7 (24.14\%) &  7 (25.93\%) &  8 (21.62\%) & 10 (32.26\%) & 32 (25.81\%) \\
% \hline
% Total       & 29 (100\%)   & 27 (100\%)   & 37 (100\%)   & 31 (100\%)   & 124 (100\%) \\
% \hline\hline
% \end{tabular}
% \begin{tablenotes}[para,flushleft]
% \fontsize{4}{1} \selectfont 
% \end{tablenotes}
% \end{threeparttable}
% \end{table}
% \end{frame}

\begin{frame}{Application Rate by Treatment Group}
\begin{center}
    \includegraphics[height= 7cm, width = 7cm]{afe_visuals/application_rates.png}
\end{center}
\end{frame}

\begin{frame}{Application Status by Treatment Group}
    \begin{center}
        \includegraphics[height= 7cm, width = 8.5cm]{afe_visuals/application_stacked.png}
    \end{center}
\end{frame}

\begin{frame}{Reasons for Denial}
    \begin{table}[htbp]\centering
\begin{tabular}{lrrr}
\toprule
Reason for Denial & Freq. & Percent & Cumulative \\
\midrule
 & 184 & 95.83 & 95.83 \\
Failure to provide proof of income & 1 & 0.52 & 96.35 \\
Failure to provide & 1 & 0.52 & 96.87 \\
Meal plan & 1 & 0.52 & 97.40 \\
Missed intake interview & 3 & 1.56 & 98.96 \\
Over income & 1 & 0.52 & 99.48 \\
Student eligibility & 1 & 0.52 & 100.00 \\
\midrule
\textbf{Total} & \textbf{192} & \textbf{100.00} & \\
\bottomrule
\end{tabular}
\end{table}
\end{frame}

\begin{frame}{Change in Food Insecurity Score by Treatment Group}
\begin{center}
    

    \includegraphics[height= 7cm, width = 8.5cm]{afe_visuals/diff_fi.png}

\end{center}
\end{frame}
\end{document}
