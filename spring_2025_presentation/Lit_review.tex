\documentclass{article}
\usepackage[margin=.75in]{geometry}
\begin{document}
\begin{itemize}
    \item (SNAP) provides monthly food-purchasing assistance for nearly 41 million low-income individuals, including 20 million children, each year
    \item Numerous studies have shown that households increase expenditures right after SNAP receipt and subsequently decrease consumption throughout the benefit month, creating periods in which many may go without food (Shapiro 2005; Tarasuk, McIntyre, and  Li 2007; Castner and  Henke 2011; Todd 2015).
    \item When SAT dates fall more than two weeks after a student’s SNAP benefit issuance date, SAT scores are six points lower for low-income students. This translates into lower four-year college attendance (Bond et al. 2022). 
    \item Observational evidence found that adolescents in India from food insecure households  received lower vocabulary, reading, maths and English scores in early adolescence. Results were robust to controls for education and health investments, parental and children's educational aspirations, and children's psychosocial skills (Aurino et al. 2019)
    \item In 2023, households in the lowest income quintile spent an average of \$5,278 on food representing 32.6 percent of after-tax income. Households in the middle income quintile spent an average of \$8,989 on food representing 13.5 percent of after-tax income.  Households in the highest income quintile spent an average of \$16,996, 8.1 percent of after-tax income
    \item Data based on surveys of store prices show that low-income households likely face slightly higher prices, by nearly 1 percent, than the national average for a given set of food items (Kaufman et al. 1997)
    \item Percent of respondents who reported at least 1 food insecurity issue: \\black: 28.7\% \\
    white: 11.2\%\\
    hispanic: 23.2\%\\
    other: 13.4\%\\
    (Lee et al. 2022)
    \item Similar results are found in CPS data with the following percent of household reporting food insecurity at some point during the year of 2023:\\
    white: 9.9\%\\
    black: 23.3\%\\
    hispanic: 11.9\% \\
    other: 12\%\\
    \item I couldn't find stats available on share of income by race but I did some back of the envelope calculations using BLS data\\ 
    White income: 87760	\\
    White food expenditures: 8388\\
    Percentage: 9.6\% \\
    Asian income: 119995 \\
    Asian food expenditures: 10527 \\
    Percentage: 8.8 \% \\
    Black income: 60788 \\
    Black food expenditures: 6052\\
    Percentage: 10\% \\
    \textbf{Note:} The white category also includes  Native Hawaiian or other Pacific Islander, American Indian or Alaska Native, as well as respondents reporting more than one race and I think Hispanic so I would guess that the number I calculated is an upper bound
    \item Quasi experimental results indicate differences in students’ math and reading performance based on the recency of SNAP benefit transfer. (Pines and Bellows 2018)
    \begin{itemize}
        \item  Controlling for race/ethnicity, gender, and grade, as well as school fixed effects the difference between SNAP-recipient and nonrecipient students is substantial: 0.36 standard deviations for math and 0.35 standard deviations for reading
        \item The average score by days since SNAP transfer ranges from the peak to trough by 17\% of a standard deviation for reading and 12\% of a standard deviation for math
    \end{itemize}
    \item In fiscal year 2023 children accounted for about 39 percent of all Supplemental Nutrition Assistance Program (SNAP) participants 
    \item 86.5 percent (114.6 million) of U.S. households were food secure throughout 2023  13.5 \% food insecure 
    
    \item SNAP racial  breakdown of enrolled citizens : \\
    White: 37 percent \\
    African American: 26 percent\\
    Hispanic: 16 percent
    Asian: 3 percent\\
    Native American: about 2 percent. 
     \textit{(About 16 percent of participants are categorized as “race unknown.”)}
     \item In 2012 food expenditures as percentage of disposable income was approximately 9.6 \% and has since grown to roughly 10.8 \% 
     \item There is a similar trend with household disposable income with pre pandemic levels peaking in 2017 at 10.54\% and most recently in 2023 hitting 11.21\%
     \item CalFresh by race:
     California population: 39,000,000\\
     Latino: 40\% $\rightarrow$ 15,600,000\\
     Calfresh: 3,048,000 \\
     Percent: 19.5\% \\
     White: 34\% $\rightarrow$ 13,260,000\\
     Calfresh: 1,086,000\\
     Percent: 8.2\%\\
     API: 16\%  $\rightarrow$ 6,240,000\\
     Calfresh: 628,000\\
     Percent: 10.1\%\\
     Black: 6\% $\rightarrow$ 2,340,000\\
     Calfresh: 483,000\\
     Percent: 10.6\%\\
     \item  398,000 people in San Diego are receiving calfresh benefits as of 2024
     
\end{itemize}


\end{document}

