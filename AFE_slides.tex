
\documentclass{beamer}
\DeclareGraphicsExtensions{.eps, .pdf,.png,.jpg,.mps,}
\usepackage{graphicx}
\usepackage{epstopdf}
\usepackage{verbatim}
\usepackage[flushleft]{threeparttable}
\usepackage{caption}
\usepackage{hyperref}
\usepackage{subfig}
\usepackage{adjustbox}
\usepackage{tablefootnote}
\usepackage{overpic}
\usepackage{changepage}
\usepackage{lmodern}
\usepackage[utf8]{inputenc}
\usepackage[T1]{fontenc}
\usepackage{dcolumn}
\usepackage{booktabs}
\usepackage{threeparttable}
\usepackage[usenames,dvipsnames]{xcolor}
\usepackage{tikz} 
\usetikzlibrary{calc, arrows.meta, intersections, patterns, positioning, shapes.misc, fadings, through,decorations.pathreplacing}
\setbeamertemplate{itemize subitem}{\color{gray}\scriptsize$\bullet$}
\definecolor{ColorOne}{named}{MidnightBlue}
\definecolor{ColorTwo}{named}{Dandelion}
\definecolor{ColorThree}{named}{Plum}
\usepackage[dvipsnames]{xcolor}
\usepackage[table]{xcolor}
\usepackage{colortbl}
\usepackage{longtable}
\usepackage{tablefootnote}

% define colors
\definecolor{mypink1}{rgb}{0.858, 0.188, 0.478}
\definecolor{grn}{rgb}{0.0, 0.5, 0.0}
\definecolor{orn}{rgb}{1.0, 0.49, 0.0}
\definecolor{lightgreen}{RGB}{220, 255, 220}
\definecolor{medgreen}{RGB}{170, 230, 170}


% indentation
\renewcommand{\indent}{\hspace*{2em}}

\setlength{\leftmarginii}{3em}
\setbeamertemplate{itemize items}[circle]
\setbeamertemplate{footline}[frame number]
\usepackage{appendixnumberbeamer}

\usepackage[style=authoryear]{biblatex}
\bibliography{biblography}
\usepackage{soul}
\bfseries
\newcommand{\Blue}{\color[rgb]{0,0,1}}
\newcommand{\hilight}[1]{\colorbox{orange}{#1}}
\usepackage{soul}
\bfseries

\newenvironment{wideitemize}{\itemize\addtolength{\itemsep}{10pt}}{\enditemize}
\newenvironment{mfignotes}{\begin{tiny}\begin{minipage}{\textwidth}\begin{tiny}\smallskip\par Notes: }
{\end{tiny}\end{minipage}\end{tiny}}
\title []{Preliminary Evidence on the Impact of Assistance and Simulated Benefits on Take-Up}

\author[Aman et al. (2025)] { Christine Aman, UC San Diego\\
Eric Floyd, UC San Diego \\
Mia Jerphagnon, UC San Diego \\
Sally Sadoff, UC San Diego}

\date{Fall, 2025}
\makeatletter
\setbeamertemplate{navigation symbols}{}
\begin{document}



\begin{frame}
 %\addtocounter{framenumber}{-1}
\titlepage
%\thispagestyle{empty}
\end{frame}

% \begin{frame}{Motivation - CUT?}
% \begin{adjustwidth}{-3em}{-2em}
%     \begin{wideitemize}
%         \item A primary obstacle to lifting people out of poverty is ensuring that those who are eligible for government assistance receive it
%         \item It is estimated 20\% to 50\% of Americans do not take up programs they are eligible for {\footnotesize (Giannarelli, 2019; IRS, 2024b; Vigil, 2022)}
%         \item Because eligible people are not enrolled in programs by default those who are seeking benefits face high administrative burden when applying
%        \item To bridge this gap research has focused on the use of light and heavy touch interventions often aimed to lower administrative burdens
%     \end{wideitemize}

% \end{adjustwidth}
% \end{frame}



\begin{frame}{Motivation}
\begin{adjustwidth}{-3em}{-2em}
    \begin{wideitemize}
        \item Large literature documents low take-up of social programs (e.g., EITC, Medicaid, SNAP): 20-50\% of eligible Americans fail to take up {\footnotesize (Giannarelli, 2019; IRS, 2024b; Vigil, 2022)}
        \item Evidence from firms suggests default enrollment would have substantial positive impacts on take-up (Madrian \& Shea, 2001; Beshears et al., 2008) 
        \item Growing evidence suggests that light-touch outreach and information about social programs may have a limited impact on take-up 
        \begin{itemize}
            \item Meta-analytic evidence estimates an average effect size of 1.4 percentage points on take-up (DellaVigna \& Linos 2020)
        \end{itemize}
        \item Higher-touch interventions with intensive assistance have mixed results 
    \end{wideitemize}
    \vspace{1em}
\end{adjustwidth}
\end{frame}


\begin{frame}{Motivation}
\begin{adjustwidth}{-3em}{-2em}
    \begin{wideitemize}
        \item Take-up of assistance appears to account for a large amount of this variation
    \end{wideitemize}
    \begin{center}
  \includegraphics[width=12cm, keepaspectratio]{afe_visuals/coef_plot_f.png}  
\end{center}
\end{adjustwidth}
\end{frame}


% \begin{frame}{Motivation}
% \begin{adjustwidth}{-3em}{-2em}
%     \begin{wideitemize}
%         \item Interventions where the enrollment assistance is offered as part of an already sought out service (closer to default enrollment) have been able to produce higher levels of takeup 
%     \end{wideitemize}

%   \includegraphics[width=12cm, height=7.5cm]{afe_visuals/coef_plot_3.png}  
% \end{adjustwidth}
% \end{frame}


% \begin{frame}{Our Study}
% % add more figures takeup, eligibility of snap nationally 
% % 
% \begin{adjustwidth}{-3em}{-2em}
% \begin{wideitemize}
%     \item We test interventions with the goal of increasing take-up of CalFresh (SNAP) 
%     \item In the US 17\% of US households with children experience food insecurity 
%     \item Nationally, roughly 1 in 5 eligible households are not enrolled in the program
%     \item Our interventions are targeted towards addressing the following barriers to enrollment:
%     \begin{enumerate}
%         \item Administrative burden $\rightarrow$ Address by providing personalized application assistance 
%         \item Underestimating program impact $\rightarrow$ Address by simulating a test enrollment using grocery cards
%     \end{enumerate}
    
% \end{wideitemize}

% \end{adjustwidth}
% \end{frame}

\begin{frame}{Our Study}
% add more figures takeup, eligibility of snap nationally 
\begin{adjustwidth}{-3em}{-2em}
\begin{wideitemize}
    \item Test interventions to increase take-up of CalFresh (SNAP/food stamps) among university students and measure impact on enrollment and student well-being
    \item $\approx$ 33\% of UCSD students are eligible for CalFresh, $\approx$ 70\% do not enroll
        \begin{wideitemize}
            \item $\approx$ 15\% of US households are eligible for SNAP, $\approx$ 20\% do not enroll
            \item Estimated 42\% of UCSD students report food insecurity compared to 13\% of US households 
        \end{wideitemize}

    \item Food insecurity affects student well-being 
     \begin{itemize}
        \item SATs being taken during the last two weeks of the SNAP benefit cycle reduces SAT scores and college enrollment (Bond et al. 2022)
    \end{itemize}
\end{wideitemize}
\end{adjustwidth}
\end{frame}

\begin{frame}{Experimental design}
% add more figures takeup, eligibility of snap nationally 
\begin{adjustwidth}{-3em}{-2em}
\begin{wideitemize}
    

    \item Test intensive personalized assistance, similar to "intensive interventions" tested in pior literature
    \item Test ``simulated benefits", which gives experience with benefits prior to and during application process
   % \item Our interventions are targeted towards addressing the following barriers to enrollment:
        \begin{itemize}\normalsize{
            \item Could increase willingness to apply (underestimate benefits, endowment effects, salience)
            \item Provides benefits during time consuming application process
            \item Allows measurement of impact of (short-term) automatic enrollment on food insecurity, mental health and academic outcomes
            %\item Underestimating program impact $\rightarrow$ Address by simulating a test enrollment using grocery cards
            }
             \end{itemize}
        \item Test combined treatments given that we cannot auto enroll students into CalFresh

             \begin{itemize}\normalsize{
            \item Simulated benefits may motivate greater take-up of assistance
            }
             \end{itemize}
       
\end{wideitemize}
\end{adjustwidth}
\end{frame}

\begin{frame}{Experimental Design}
% "Grocery Card": auto enrollment mimic 
\begin{adjustwidth}{-3em}{-2em}
    \begin{wideitemize}
        \item {\bf{Control:}} Received an email stating that we had determined they were eligible to apply for CalFresh linking a webpage about enrolling
        \item {\bf{Assistance:}} Assistants reached out to the participants receiving assistance via email and offer to help with the process of signing up via zoom or email 
\item \textbf{Simulated Benefits:} Participants received a \$100 grocery card to the grocery store of their choice 
\item \textbf{Assistance + Simulated Benefits:} The final treatment group received both the grocery card and assistance treatments

\end{wideitemize}
\end{adjustwidth}

\end{frame}

\begin{frame}{Population and Timeline}

 \begin{itemize}
        \item Our sample consists of 125 UCSD students who we deemed to be likely eligible for CalFresh 
    \end{itemize}
\vspace{1em}
\begin{adjustwidth}{-2em}{-3em}
\tikzstyle{descript} = [text = black,align=left, minimum height=1.8cm, align=center, outer sep=0pt,font = \footnotesize]
\tikzstyle{activity} =[align=left,outer sep=1pt]

\begin{tikzpicture}[very thick, black]

%% Coordinates
\coordinate (O) at (-1.5,0); % Origin
\coordinate (P1) at (3,0);
\coordinate (P2) at (8,0);
\coordinate (P3) at (11,0);
\coordinate (F) at (13,0); %End

\coordinate (E2) at (3,0); %Event

%% Filled regions
\fill[color=ColorOne!20] rectangle (0) -- (P1) -- ($(P1)+(0,1)$) -- ($(0)+(0,1)$); 

\fill[color=ColorTwo!20] rectangle (P1) -- (P2) -- ($(P2)+(0,1)$) -- ($(P1)+(0,1)$); % Work
\fill[color=ColorThree!20] rectangle (P2) -- (P3) -- ($(P3)+(0,1)$) -- ($(P2)+(0,1)$); % Current work

%% Text inside filled regions
\draw ($(P1)+(-1.5,0.5)$) node[activity,ColorOne] {Enrollment \&\\ Baseline Survey};
\draw ($(P2)+(-2.5,0.5)$) node[activity,ColorTwo] {Outreach / Assistance};
\draw ($(P3)+(-1.5,0.5)$)  node[activity, ColorThree] {Endline Data \\Collection};




\draw [decorate,decoration={brace,amplitude=6pt}]($(E2)+(0,1.2)$) -- ($(E2)+(1,1.2)$) node [black,midway,above=6pt] {\scriptsize Grocery Card Selection / Disbursement};




\foreach \x in {0, 3, 8, 11}
\draw(\x cm,3pt) -- (\x cm,-3pt);
%% Labels
\foreach \i \j in {0/week 1, 3/week 4,8/week 9, 11/week 11,}{
\draw (\i,0) node[below=3pt] {\j} ;
}

\end{tikzpicture}
\end{adjustwidth}
\end{frame}


% \begin{frame}{Population}
% \begin{adjustwidth}{-2em}{-2em}
%     \begin{wideitemize}
%         \item Our sample consists of 125 UCSD students who we deemed to be likely eligible for CalFresh 
%     \end{wideitemize}
% \begin{center}
%     \includegraphics[width=7cm, height=6cm]{afe_visuals/food_ins_ucsd.png}
% \end{center}
% \begin{wideitemize}
%     \item As of fall 2019, only 22\% and 29\% of eligible UC undergraduates and graduates were enrolled in CalFresh 
% {\footnotesize(Rothstein et al., 2024)}
% \end{wideitemize}

\end{adjustwidth}
\end{frame}


\begin{frame}{Sample Characteristics}
\begin{threeparttable}
\begin{tabular}{@{\extracolsep{5pt}}lcc}
\\[-1.8ex]\hline \hline \\[-1.8ex]
 & Mean & Interpretation \\ \hline \\[-1.8ex] 
Female                  & 0.694 & Proportion female \\
White                   & 0.211 & Proportion identifying as White \\
Black                   & 0.041 & Proportion identifying as Black \\
Hispanic                & 0.341 & Proportion identifying as Hispanic \\
Asian                   & 0.463 & Proportion identifying as Asian \\
Financial Aid           & 0.532 & Proportion receiving aid \\
Food Insecurity         & 2.492 & USDA, (0--10); 2--4: Low food security \\
Depression              & 19.75 & CES-D, (0--60); 15--21: Mild depression \\
Stress                  & 19.67 & PSS, (0--40); 14--26: Moderate stress \\
Anxiety                 & 7.815 & GAD-7, (0--21); 5--9: Mild anxiety \\
\hline \hline
\end{tabular}
\begin{tablenotes}[flushleft]
\footnotesize
\item Food insecurity questions are asked in reference to the previous month
\end{tablenotes}
\end{threeparttable}
\end{frame}


\begin{frame}{Assistance Take-up by Treatment}
    \begin{center}
        \includegraphics[width = \textwidth, \keepaspectratio]{afe_visuals/assistance_takeup.png}
    \end{center}
\end{frame}


% \begin{frame}[t]{County Data: Application Rate by Treatment}
% \vspace*{1cm}
% \tiny
% \centering
% \begin{table}[htbp]
% \centering
% \begin{threeparttable}
% \caption{Application Rate by Treatment}
% \label{tab:outcomes}
% \begin{tabular}{l*{4}{c}c}
% \hline\hline
% & \multicolumn{1}{c}{Group 0} & \multicolumn{1}{c}{Group 1} & \multicolumn{1}{c}{Group 2} & \multicolumn{1}{c}{Group 3} & \multicolumn{1}{c}{Total} \\
% & \multicolumn{1}{c}{}  & \multicolumn{1}{c}{}  & \multicolumn{1}{c}{}  & \multicolumn{1}{c}{}  & \multicolumn{1}{c}{} \\
% \hline
% Did not apply           & 22 (75.86\%) & 20 (74.07\%) & 29 (78.38\%) & 21 (67.74\%) & 92 (74.19\%) \\
% Applied           &  7 (24.14\%) &  7 (25.93\%) &  8 (21.62\%) & 10 (32.26\%) & 32 (25.81\%) \\
% \hline
% Total       & 29 (100\%)   & 27 (100\%)   & 37 (100\%)   & 31 (100\%)   & 124 (100\%) \\
% \hline\hline
% \end{tabular}
% \begin{tablenotes}[para,flushleft]
% \fontsize{4}{1} \selectfont 
% \end{tablenotes}
% \end{threeparttable}
% \end{table}
% \end{frame}

\begin{frame}{CalFresh Application Rate by Treatment Group}
\begin{center}
    \includegraphics[width = \textwidth]{afe_visuals/application_rate.png}
\end{center}
\end{frame}

\begin{frame}{Application Status by Treatment Group}
\vspace{-.5cm}
    \begin{center}
        \includegraphics[width = \textwidth, \keepaspectratio]{afe_visuals/application_stacked.png}
    \end{center}
\end{frame}



\begin{frame}{Change in Food Insecurity Score by Treatment Group}
\begin{center}
    

    \includegraphics[width = 10cm, \keepaspectratio]{afe_visuals/change_fi.png}

\end{center}
\end{frame}

\begin{frame}{Application Submission Timing by Treatment}
    \begin{center}
         \includegraphics[height= 8cm, width = 10.5cm]{afe_visuals/time_application.png}
    \end{center}
\end{frame}


% \begin{frame}{Application Submission Timing by Treatment}
%     \begin{center}
%          \includegraphics[ width = 10.5cm, \keepaspectratio]{afe_visuals/time_application_raw.png}
%     \end{center}
% \end{frame}


\begin{frame}{Conclusions and Next Steps}
\begin{adjustwidth}{-2em}{-2em}
    \begin{wideitemize}
    \item We find suggestive evidence that the combined simulated benefits and assistance increased take-up of assistance and benefits
    \item Running a second wave at the start of the fall quarter (begins September 25) with subsequent waves in each academic quarter
    \item We will expand our analysis to look at academic outcomes and well-being 
    \end{wideitemize}
    \vspace{1em}
    \begin{center}
        Please reach out to me (caman@ucsd.edu) with any feedback or suggestions 
    \end{center}
\end{adjustwidth}
\end{frame}

%%%%%%%%%%%%%%%%%%%%%%%%%%%%%%%%%%%%%%%%%%%%%%%%%%%%%%%%%%%%%%%%%
\begin{frame}{Reasons for Denial}
    \begin{table}[htbp]\centering
\begin{tabular}{lrrr}
\toprule
Reason for Denial & Freq. & Percent & Cumulative \\
\midrule
 & 184 & 95.83 & 95.83 \\
Failure to provide proof of income & 1 & 0.52 & 96.35 \\
Failure to provide & 1 & 0.52 & 96.87 \\
Meal plan & 1 & 0.52 & 97.40 \\
Missed intake interview & 3 & 1.56 & 98.96 \\
Over income & 1 & 0.52 & 99.48 \\
Student eligibility & 1 & 0.52 & 100.00 \\
\midrule
\textbf{Total} & \textbf{192} & \textbf{100.00} & \\
\bottomrule
\end{tabular}
\end{table}
\end{frame}
\begin{frame}{Participants}
% ADD IN THAT THIS IS WAVE 1
% ADD IN SAMPLE SIZE WHICH IS JUST THOSE WHO FILLED OUT ROI
% ADD TIMELINE BELOW
\begin{adjustwidth}{-3em}{-3em}
\begin{wideitemize}
    \item We enrolled only students who we determined to be eligible to \textit{apply} for CalFresh (who were not already currently enrolled)
    \begin{wideitemize}
        \item To be eligible to apply students must be U.S. citizens, be at least 18 and meet one of the following: \begin{enumerate}
            \item Approved and accepted a Federal Work Study award
            \item Received a TANF funded CalGrant A or B award
            \item Receiving California Work Opportunity and Responsibility to Kids (CalWORKs) or Tribal TANF
            \item Employed with an on campus job
            \item In school program on LPIE list
        \end{enumerate}
    \end{wideitemize}
    \item Being eligible to apply for CalFresh does not necessarily mean that one is eligible for benefits 
\end{wideitemize}
\end{adjustwidth}
\end{frame}


%     \begin{wideitemize}
%         \item At UCSD, approximately 42\% of the undergraduate and 35\% of the graduate students have reported experiencing food insecurity
%         \begin{itemize}
%             \item The need is even more acute for underserved and underrepresented minority and Pell Grant-eligible  students, with rates rising to 53\% and 51\%, respectively (Carillo, 2024)
%         \end{itemize}
%         \item As of fall 2019, only 22\% and 29\% of eligible UC undergraduates and graduates were enrolled in CalFresh (California's version of the SNAP)
% (Rothstein et al., 2024).
%     \end{wideitemize}


%     \begin{frame}{Our Study}
% % interest in actual enrollment process and the effects of receiving benefits 
% % Two treatments 1 resolves the problem of administrative burden, provides relief while enrollment process stretches on, gives us a sense of what effect of automatic enrollment 
% % link grocery card and link to automatic enrollment 
% \begin{adjustwidth}{-2em}{-2em}
%     \begin{wideitemize}
%         \item Conduct a field experiment to test the impacts of personalized outreach and simulated automatic enrollment 
%         \item These interventions address two primary barriers to CalFresh enrollment:
%         \begin{itemize}
%             \item Many students are deterred from enrolling in CalFresh due to its complicated application process 
%             \item Students may not fully understand the benefits of
% being on CalFresh
%         \end{itemize}
%         \item Examine the impact of personal outreach and step-by-step assistance with the enrollment process and of providing students with grocery cards that mimic the purchasing power of CalFresh EBT cards
%     \end{wideitemize}
% \end{adjustwidth}
% \end{frame}


% \begin{frame}{Sample}
% %BALANCE TABLE HERE 
% \begin{adjustwidth}{-3em}{-3em}
% The baseline survey collected information on:
%     \begin{wideitemize}
%         \item Eligibility to apply for Calfresh
%         \item Food Insecurity 
%         \begin{itemize}
%             \item Measured by the U.S. Adult Food Security Survey Module
%         \end{itemize}
%         \item Mental health 
%         \begin{itemize}
%             \item Measured by GAD-7 and CES-D
%         \end{itemize}
%         \item Consumption of major food groups 
%         \begin{itemize}
%             \item Reported consumption of major food groups from the Dietary Guidelines for Americans, 
%             \item Whether participants wants to increase, decrease, or maintain consumption of major food groups
%         \end{itemize}
%     \end{wideitemize}
% The endline survey collected the same information with the following changes
% \begin{wideitemize}
%     \item No eligibility information was collected
% \end{wideitemize}
% \end{adjustwidth}
% \end{frame}





% Say CalFresh is SNAP / Food Stamps / EBT 
% Tell more of a story 
% Tell people a bit of what is here and what should be learned 
% Our Study: 
% emphasize that student population is surprisingly food insecure population - higher need than expected and much lower take up
% more than national average but a full 70% do not enroll - big opportunity to help students 
% 
% split our study into two 
% add in stuff about food insecurity and academics / mental health -- why do we care 
% Could show other schools 
% add percent of of food insecurity 
% Make interpretation more clear 
% emphasize that SE is motivated by the default enrollment 
% explain why it might increase willingness to apply 

% Just have mean and p-value 
% add reference distribution 
% Get rid of N 
% add reference range 
% remove p values and add footnote that no significant (add sd below)
% 

%assistance take up:
% get rid of raw numbers 
% remind people of two treatment groups and what they are 
% flip order of bars 
% fix ugly colors ):
% Point out that people are more receptive -- say that 

%GET RID OF Assistance Take-up Among CalFresh Applicants

%LOOK AT DATE OF APPLICATION **cumulative applications as of given dates - jump for control at start 
%In progress as a mechanism is interesting 
% Emphasize small sample -- change in food insecurity scores 
%Note how food insecurity questions are asked and that it is self reported 
% could be noise or could be changes in perception of food insecurity 

%Conclusions / Takeaways added to next steps - what mechanisms 
%In future waves is there anything we should consider 
% Some theory on simulating enrollment interacting with assistance 
% Ideally we would have default enrollment but given we don't have that people may still need assistance so you could be motivated by SE but not know how to proceed 
% Distinciton between simulated benefits instead of simulated enrollment  
% Put in that take up of assistance is generally low highlight link between low take up and low take up benefits 
% Cange to percentage pf control in lit review 
% change y axis of application status of treatment group 
% Get rid of standard error bars and put them in button 

\end{document}